\documentclass{article}
\usepackage[utf8]{inputenc}
\usepackage[letter, total={7in, 9.5in}]{geometry}
\usepackage{hyperref}
\usepackage{fancyhdr}
\renewcommand{\headrulewidth}{0pt}

\RequirePackage{mathpazo}

\hypersetup{
    colorlinks=true,
    linkcolor=red,
    filecolor=magenta,      
    urlcolor=blue,
    pdftitle={Overleaf Example},
    pdfpagemode=FullScreen,
    }
\urlstyle{same}


\newcommand{\Name}[1]{
\noindent
\huge{#1}
\vspace{0.05in}
%\par
\normalsize
}

\newcommand{\ContactInfo}[5]{
\noindent
#1 #2 #3 $\mid$ #4 $\mid$ #5
%\par
}


\newcommand{\Class}[4]{{\bf #1:} #2 (#3), ({\em #4})}

\newcommand{\ClassNoSkills}[3]{{\bf #1:} #2 (#3)}

\newcommand{\ClassInProgress}[3]{{\bf #1:} #2 ({\em#3})}

\newcommand{\ClassInProgressNoSkills}[2]{{\bf #1:} #2}

\newcommand{\cvhline}{
\vspace{0.1in}
\hline
\vspace{-0.1in}}

\newcommand{\cvhlineb}[]{
\hline
\vspace{-0.1in}}


\newcommand{\secvspace}{\vspace{-0.1in}}

\newcommand{\WorkExpURL}[1]{\small(\url{#1})\normalsize}

\pagenumbering{gobble}


\begin{document}
\pagestyle{fancy}
\fancyhead{}
\fancyfoot[R]{Updated Sep. 20, 2022}
\Name{Giuseppe Perona}\large{ Berkeley \& Altadena, Calif. $\mid$ giuseppe.perona@berkeley.edu}\normalsize
\cvhlineb
\section*{Education}
\secvspace
\begin{description}
    %\itemsep-0.15cm
    \item[2020-2024 (expected):] Computer Science B.A., City Planning Minor -- UC Berkeley, Berkeley, Calif.\quad GPA: 3.7
    \item[2016-2020:] St. Francis High School, La Ca\~{n}ada, Calif.
    \item[Relevant Coursework:]%\quad\\
    {\em Previously Taken -- } 
    \Class{CS 61A}{Structure and Interpretation of Computer Programs}{A-}{Python, Scheme, SQL},\quad
    \ClassNoSkills{MATH 54}{Linear Algebra and Differential Equations}{B+},\quad
    \Class{STAT 20}{Introduction to Probability and Statistics}{P}{R},\quad
    \Class{CS 61B}{Data Structures}{A-}{Java, Git},\quad
    \Class{EECS 16A}{Designing Information Devices and Systems I}{B+}{Python, NumPy, Linear Algebra},\quad
    \Class{DATA 8}{Foundations of Data Science}{B+}{Python, NumPy},\quad
    \ClassNoSkills{CS 70}{Discrete Mathematics and Probability Theory}{B+},\quad
    \ClassNoSkills{MATH 53}{Multivariable Calculus}{A},\quad
    \Class{GEOG 80}{Introduction to Geospatial Technologies}{A}{ArcGIS},\quad
    \ClassNoSkills{ECON 101B}{Macroeconomic Theory – Macro, Math Intensive}{A-},\quad 
    \ClassNoSkills{STAT 134}{Concepts of Probability}{A-},\quad
    \Class{EECS 16B}{Designing Information Devices and Systems II}{A}{Linear Algebra, Differential Equations, Python, NumPy}
    
    {\em In Progress -- }
    \ClassInProgressNoSkills{EECS 127}{Optimization Models in Engineering}, \ClassInProgress{CIVENG 263/CY PLAN 257}{Scalable Spatial Analytics/Data Science for Human Mobility}{NetworkX, sklearn}
    \ClassInProgress{DATA 100}{Principles and Techniques of Data Science}{Python, pandas, NumPy, Matplotlib, sklearn, SQL}, 
    \ClassInProgressNoSkills{CY PLAN 110}{Introduction to City Planning}

    {\em Planned -- }
    \ClassInProgressNoSkills{CS 189}{Machine Learning}, \ClassInProgressNoSkills{CS 170}{Efficient Algorithms and Intractible Problems}, \ClassInProgressNoSkills{CIVENG 264}{Behavioral Modeling for Engineering, Planning, and Policy Analysis}
\end{description}
\cvhlineb
\section*{Skills}
\secvspace
\begin{description}
    %\itemsep-0.15cm
    \item[Programming Languages:] Python, Java, SQL, R, Scheme
    \item[Tools:] NumPy, pandas, Geopandas, Matplotlib, Scikit-learn, \LaTeX, ArcGIS, Git
    \item[Spoken Languages:] {\em Native:} English, Italian, {\em Fluent:} Spanish
\end{description}
\cvhlineb
\section*{Work Experience}
\secvspace
\begin{description}
    %\itemsep-0.15cm
    \item[Summer 2022, Dantercepies, Selva Val Gardena, Italy:]\WorkExpURL{http://www.dantercepies.it/en/} Analyzed skier flow data from Val Gardena and found beahvior patterns via the K-Means algorithm, and devised a metric to quantify the impact of queues on travel times.
    \item[Summer 2019 -- North South Rail Link, Boston:] \WorkExpURL{http://www.northsouthraillink.org} Discussed strategy for how to build public enthusiasm for the project. Wrote a report on causes and prevention of railway accidents in light of two major derailments on the MBTA Subway.
    \item[July 2018 -- Rail Delivery Group (RDG), London: ]\WorkExpURL{www.raildeliverygroup.com} Work experience assignments included writing a report on the history of rail in the UK, the place of rail in the future, and the impact of the closure of peripheral, rarely used lines. Visited railway depots, as well as Atos \WorkExpURL{https://atos.net/en/}, where UK rail operations are analyzed.

    \item[June 2018 -- Dantercepies:] Assisted chief of operations in carrying out seasonal safety tests, replaced a 400KW electric motor with a crew of two other mechanics for a monocable detachable gondola lift.
\end{description}
\cvhlineb
\section*{Projects \& Extracurriculars}
\secvspace
\begin{description}
    %\itemsep-0.15cm
    \item [Cal Transpo/Institute of Transportation Engineers]
    \item[Fall 2021, GEOG 80:] Used Python, and the Pandas, Geopandas, and Matplotlib libraries to analyze BART ridership data over the COVID-19 pandemic, and examine how different factors (local income, population density, etc.) affected ridership. Used ArcGIS to display findings, and presented in an ArcGIS Story Map. {\em Received grade A for the project}
    \item[Summer 2021, Personal Project:] Built a model of a city in Java to compare the efficiency of transportation systems with different block shapes, eg. hexagonal vs. rectangular blocks. Created a new coordinate system to use with a hexagonal grid.
    \item[Spring 2021, CS 61B:] Developed a custom version of Git in Java. Supports basic commands: commits, log, checkout, merge. {\em Received full credit for the project}
\end{description}

\end{document}
